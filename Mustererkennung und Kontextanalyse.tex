 \documentclass{article} %A4
\usepackage[a4paper,left=1.9cm, right=2.1cm,top = 1.2cm,bottom=2.3cm]{geometry}
\usepackage[utf8]{inputenc}%Umlaute
\usepackage[ngerman]{babel} %Texttrennung
\usepackage{graphicx}	%Grafiken
\usepackage{amssymb}
\usepackage{amsmath}
\usepackage{amsthm}
\usepackage{url}
\usepackage{listings}
 \usepackage{color}
\usepackage{hyperref}
\usepackage{framed}
\usepackage{algpseudocode}
\usepackage{tikz}

\usepackage[labelformat=empty]{caption}
\title{Zusammenfassung - Mustererkennung und Kontextanalyse}
\author{
	Andreas Ruscheinski
}
\theoremstyle{definition} 
\newtheorem*{Abtasttheorem}{Abtasttheorem} 

\begin{document}
\maketitle
\begin{framed}Korrektheit und Vollständigkeit der Informationen sind nicht gewährleistet.
Macht euch eigene Notizen oder ergänzt/korrigiert meine Ausführungen!
\end{framed}
\setcounter{tocdepth}{1}
\tableofcontents

\section{Überblick Klassifikation}
	\subsection{Einführendes Beispiel}
	\begin{itemize}
		\item Ziel: Bestimmung von Fischen auf der Basis von Kamerainformationen
		\item Verwendung der Kamera zum Merkmale (Features) des aktuellen Fisches zu bestimmen: Länge, Helligkeit, Breite
		\item Annahme: Die Modelle (Beschreibung) der Fischen unterscheiden sich
		\item Klassifikation: finde zu gegebene Merkmalen das am besten passende Modell (Welche Beschreibung der Fische passt am besten zu den aktuellen Fisch)
	\end{itemize}
	\subsection{Mustererkennungssysteme}
		\begin{description}
			\item[1) Sensing:] Erfassung der Umwelt mittels Sensoren, z.B: Kamera, Bewegungssensoren, Mikrophon, RFID-Lesegerät, Problem: Eigenschaften und Begrenzungen (Bandbreite, Auflösung, Empfindlichkeit, Verzerrung, Rauschen, Latenz, \dots)des Sensors beeinflussen Problemschwierigkeit 			
			\item[2) Segmentation:] Identifikation der einzelnen Musterinstanzen (Identifikation der für unser Problem relevanten Daten), z.B: Fisch aufn Fließband
			\item[3) Merkmalsberechnung:] Bestimmung von Features, gesucht sind dabei Features welche eine Diskrimierungsfähigkeit haben (Zwischen zwei gleichen Klassen ähnlicher Wert, Zwischen zwei verschiedenen Klassen großer Werteunterschied) und Invariant gegenüber Signaltransformationen (Rotation, Translation, Skalierung, perspektivische Verzerrung) sind; Problem: Wie kann ich aus einer großen Auswahl an Merkmalen die besten geeigneten finden?
			\item[4) Klassifikation:] Annahme: Modelle (Grundlegende Eigenschaften) der Klassen (verschiedene Fische) unterscheiden sich; Ziel: Zuweisung von Probleminstanzen aufgrund deren Merkmale zu der am besten entsprechenden Klasse			
			\item[5) Nachbereitung:] Entscheidung auf Basis Klassifikation, Problem: Wie können wir Kontextinformationen nutzen? Können wir verschiedene Klassifikatoren zusammen nutzen? 
		\end{description}
	\subsection{Entwurf von Mustererkennungssystemem}
	\begin{description}
		\item[1) Daten sammlen:] Wie kann man wissen, wann eine Menge von Daten ausreichend groß und repräsentativ ist, um das Klassifikationssystem zu trainieren und zu testen? 
		\item[2) Merkmale bestimmen:] Gesucht: Einfach zu extrahierende Merkmale mit hoher Diskriminativität, Invariant gegenüber irrelevanten Transformationen, unempfindlich gegenüber Rauschen, Problem: Wie kann ich A-priori-Wissen nutzen?
		\item[3) Modell auswählen:] Wie erkennt man, wann ein Modell sich in der Klassifikation signifikant von einem anderen Modell – oder vom wahren Modell – unterscheidet? Wie erkennt man, dass man eine Klasse von Modellen zugunsten eines anderen Ansatzes ablehnen sollte? Versuch-und-Irrtum oder gibt es systematische Methoden?
		\item[4) Klassifikator trainieren:] Verwende gesammelte Daten, um Parameter des Klassifikators zu bestimmen
		\item[5) Klassifikator evaluieren:] Wie bewertet man die Leistung? Wie verhindert man Overfitting/Underfitting?
	\end{description}
	Für weitere Informationen sind folgende Referenzen zu konsultieren: \cite[S. 3-16]{dudaPattern}
\section{Grundlagen Signalverarbeitung}
	\subsection{Digitalisierung}
		In diesem werden die Grundlagen der Digitalen Signalverarbeitung beschrieben. In der Digitalen Signalverarbeitung werden Methoden und Techniken behandelt welche aus den anlogen Sensorwerten eine digitale Information erstellen.\\
		Herkömmliche Sensoren liefen aktuelle Werte anhand von Spannungen (Bsp: Drucksensor: $1g = 0,1V$). Dies analoge Signale werden im ersten Schritt zeitlich diskretisiert d.h. die kontinuierlichen Signale werden durch Abtastung angenährt. Im nächsten Schritt werden die abgetasten Werte diskretisiert (Diskretisierung der Amplituden) d.h. jeden abgetasten Wert wird ein Wert digitaler Wert zugewiesen.\\
		Unter Abtastung versteht man die erhebung eines Wertes zu einem Zeitpunkt. Die Häufigkeit der Abtastung wird in Herz (Hz) angegeben d.h. Abtastung mit 10 Hz enspricht 10 maliges abstasten des analogen Signales innerhalb von einer Sekunde.\
		Die Diskretisierung der Amplitude erfolt mittels einem Analog-Digtal-Wandler der entsprechend von Grenzwerten (1/2 Spannung, 1/4 Spannung, 1/8 Spannung) ensprechende Bits setzt und diese Information ausgibt.
		\subsubsection{Dithering}
		Ein Problem bei der Diskretisierung der Amplitude ergibt sich dadurch das bei einem sehr geringen Sensorwerte ensprechende Bits nichts gesetzt werden. Dies hat zur Folge das Informationen verloren gehen. Um dies zu verhindern wird Zufallsrauschen auf den aktuellen Sensorwert addiert. Dadurch wird der Grenzwert manchmal überschritten. Dadurch nährt sich der Erwartungswert den Realwert an. 
		\subsubsection{Abtastung}
		\begin{Abtasttheorem}
			Ein Signal nur dann kann korrekt abgetastet werden, wenn es keine
			Frequenzanteile enthält, die oberhalb der halben Abtastrate liegen.\\
			$f_{max} \leq \frac{1}{2}f_{sample}$
		\end{Abtasttheorem}
		Aus dem Abtasttheorem folgt wenn wir ein Signal mit $f_{max}$ korrent abzutasten müssen wir dieses Signal mit einer Frequenz von $2*f_{max}$ abtasten. 
	\subsection{Linares System}
	\begin{itemize}
		\item $f$ ist linear gdw. $f(c*(a+b)) = c*(f(a) + f(b)$ d.h. wenn f linear ist können wir die Konstante herausziehen und die einzelnen Bestandteile untersuchen
		\item Homogenität: $f(c*a) = c*f(a)$ d.h. wir können Konstanten aus unseren Berechnungen ausschließen und unser Ergebnis mit der Konstante multiplizieren
		\item Additivität: $f(a+b) = f(a)+f(b)$ d.h. wir können die zusammengesetzte Argumente von f getrennt betrachten und unsere Teilergebnisse addieren
	\end{itemize}



\bibliography{library}
\bibliographystyle{plain}

\end{document}